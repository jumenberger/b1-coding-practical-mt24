\documentclass[hidelinks]{article}
%
%%%%%%%%%%%%%%%%%%%%%%%%%%%%%%%%%%%%%%%%%%%%%%%%%%%%%%%%%%%%%%%
% START CUSTOM INCLUDES & DEFINITIONS
%%%%%%%%%%%%%%%%%%%%%%%%%%%%%%%%%%%%%%%%%%%%%%%%%%%%%%%%%%%%%%%
%
\usepackage{parskip} %noident everywhere
\usepackage{hyperref} % Show hyperlinks - claudio
\hypersetup{
    colorlinks = true
    linkcolor = blue
    urlcolor = red
    }
\usepackage{fancyhdr, lastpage} % Headers and footers
\pagestyle{fancy}
\lhead{2024}
\rhead{Claudio Vestini \& Idris Kempf} % hidden if full page
\cfoot{Page \thepage\ of \pageref{LastPage}}
\usepackage{subcaption}
\usepackage{tabto}
%
%%%%%%%%%%%%%%%%%%%%%%%%%%%%%%%%%%%%%%%%%%%%%%%%%%%%%%%%%%%%%%%
% END CUSTOM INCLUDES & DEFINITIONS
%%%%%%%%%%%%%%%%%%%%%%%%%%%%%%%%%%%%%%%%%%%%%%%%%%%%%%%%%%%%%%%
%
\pdfobjcompresslevel=0
%
\title{Submarine Mission Report}
\author{Claudio Vestini}
\date{October 2024}
\begin{document}
\maketitle
%
\section{Introduction}
This brief report will concern the B1 submarine coding practical.
Initial steps taken before starting development:
\begin{enumerate}
    \item Create and activate a virtual environment with the given requirements (numpy, matplotlib and pandas packages)
    \item Fork project repository to my GitHub account
    \item Set up a .env file to add local packages onto python PATH
    \item Make sure running files do not give any errors before branching off `main'
\end{enumerate}
%
\section{Extracting Mission Data}
The first step was to obtain mission data from the given .csv file.
\newline
I started this by creating a new branch, and modifying the Mission class inside the new branch. I implemented a new classmethod to extract each column of the mission.csv file into a separate variable, and then return the data as an instance of the Mission class.
\newline
The next step was to test the new functionality by making use of the Trajectory class's plotting methods. Once I made sure the new method was working correctly, I merged the branch back into `main', and deleted the unused branch.
%
\section{Controller Implementation}
%
\end{document}